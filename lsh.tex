\documentclass{beamer}
\usetheme{umbc4}    
\usepackage[english]{babel}
\usepackage[latin1]{inputenc}
\usepackage{times}
\usepackage[T1]{fontenc}
\usepackage{amssymb}

\newcommand{\si}{\mathrm{sim}}
\newcommand{\dist}{\mathrm{dist}}
\newcommand{\lsh}{\mathrm{LSH}}



\newcommand{\mba}{\mathrm{MBA}}
\newcommand{\lst}{\mathrm{LST}}
\newcommand{\bfs}{\mathrm{BFS}}
\newcommand{\amgm}{{\mathrm{AM-GM}}}
\newcommand{\lp}{\mathrm{LP}}
\newcommand{\lr}{\mathrm{LR}}
\newcommand{\sig}{\mathrm{sign}}

\newcommand{\leqs}{\leqslant}
\newcommand{\geqs}{\geqslant}
\newcommand{\tcg}[1]{\textcolor{green}{#1}}
\newcommand{\tcb}[1]{\textcolor{blue}{#1}}
\newcommand{\tcr}[1]{\textcolor{red}{#1}}


\newcommand{\opt}{\mathrm{OPT}}
\newcommand{\eps}{\epsilon}
\newcommand{\tum}{\mathrm{TUM}}
\newcommand{\np}{\mathrm{NP}}
\newcommand{\dtime}{\mathrm{DTIME}}
\newcommand{\mis}{\mathrm{MIS}}
\newcommand{\maxcut}{{\tt MAXCUT}}

\newcommand{\calS}{{\cal S}}
\newcommand{\calA}{{\cal A}}
\newcommand{\calB}{{\cal B}}
\newcommand{\calC}{{\cal C}}
\newcommand{\calD}{{\cal D}}
\newcommand{\calE}{{\cal E}}
\newcommand{\calF}{{\cal F}}


\newcommand{\calT}{{\cal T}}
\newcommand{\calR}{{\cal R}}


\newcommand{\mstaff}{{\tt MaxStaff}}
\newcommand{\eat}[1] {}


\newcommand{\alrt}[1]{{\color{alert} #1}}
\newcommand{\alrtl}[1]{{\color{alert}\large #1}}
\newcommand{\alrtL}[1]{{\color{alert}\Large #1}}
\newcommand{\struc}[1]{{\color{structure} #1}}
\newcommand{\strucL}[1]{{\color{structure}\Large #1}}
\newcommand{\strucl}[1]{{\color{structure}\large #1}}


\newcommand{\paintblue}[1]{$\tikz[baseline]{
            \node[fill=blue!20,anchor=base] (t1)
            {$ #1$ };
        }$ }

\newcommand{\paintred}[1]{$\tikz[baseline]{
            \node[fill=red!20,anchor=base] (t1)
            {$ #1$ };
        } $}

\newcommand {\lovasz}{{Lov\'{a}sz}}

\newcommand{\textred}[1]{{\textcolor{red}{#1}}}
\newcommand{\textblue}[1]{{\textcolor{blue}{#1}}}


\usepackage{tikz,tkz-berge}
\setbeamertemplate{background canvas}[vertical shading][bottom=green!20,top=yellow!30]

\title[CODS, Mar. 16, 2016] % (optional, use only with long paper titles)
{Fast Detection of Near Duplicates}

\author{Sambuddha Roy, LinkedIn}

\begin{document}

\begin{frame}
  \titlepage
\end{frame}

\begin{frame}
\frametitle{The duplicate detection problem.}
\begin{itemize}
\item Why are duplicates problematic? 
\pause
\item Imagine if...
\pause
\item Youtube had multiple copies of each video!
\begin{itemize}
\item Someone copies the cat video that \textit{you} uploaded!
\end{itemize}
\pause
\item Duplicate content confuses search engines.
\begin{itemize}
\item What happens to the pageRank of the page?
\end{itemize}
\end{itemize}
\end{frame}

\begin{frame}
\frametitle{How do you catch duplicates?}
\begin{itemize}
\item Some form of hashing...
\item Different items go to different hashes (collision resistant).
\item Group by hashes - each group corresponds to a distinct element.
\end{itemize}
\end{frame}

\begin{frame}
\frametitle{How about near-duplicates?}
\begin{itemize}
\item Now imagine if...
\pause
\item Someone made a profile on Facebook, very similar to yours. 
\item copied profile pictures, \pause connected to the same friends, \pause indicated same interests \pause etc.
\pause 
\item Identity theft!
\pause
\item Other use cases: plagiarism, etc. 
\color{blue} Here, put in a picture of "spot the difference"
\end{itemize}
\end{frame}


\begin{frame}
\frametitle{Essential Problem: Nearest Neighbor Search}
\begin{itemize}
\item There is a database of items $\cal{D}$. 
\item ...and a query $q$ arrives. 
\pause
\item Given the query $q$, find the nearest neighbors of $q$ in the database $\cal{D}$. 
\pause
\item Often, we just want the $k$ nearest neighbors (k-NN problem).
\end{itemize}
\end{frame}

\begin{frame}
\frametitle{Nearness?}
\begin{itemize}
\item Nearness only makes sense in the presence of a {\em distance measure}.
\pause 
\item So, similarity measures between objects/items...
\item i.e. {\em featurize} items \pause as vectors $\in \mathbb{R}^n$ or in $\{0, 1\}^n$. 
\pause
\item And consider some distance/similarity measure between these vectors. 
\end{itemize}
\end{frame}

\begin{frame}
\frametitle{Typical Distance/Similarity Measures}
\begin{itemize}
\item Jaccard similarity: $J(A, B) = \frac{|A \cap B|}{|A \cup B|}$
\pause
\item Cosine Similarity etc.: $\si(a, b) = \frac{a \cdot b}{|a||b|}$
\pause
\item Hamming Distance
\item $\ell_p$ for $p \in \left[0, 2\right]$.
\item Tanimoto distance, Mahalanobis distance, etc. 
\pause
\item It's typical to relate the distance and similarity measures as $\dist(a, b) = 1 -\si(a, b)$.
\end{itemize}
\end{frame}

\begin{frame}
\frametitle{Back to what we want...}
\begin{itemize}
\item Given a \textcolor{red}{\em query} object $q$, we would want to retrieve all the \textcolor{green}{\em database} items that are ``similar'' to $q$.
\pause
\item A search by pairwise comparisons between $q$ and all the items in the database may become too costly.
\end{itemize}
\end{frame}


\begin{frame}
\frametitle{Another use-case...}
\begin{itemize}
\item If we want to ``cluster'' the items in the database according to their similarities/distances. 
\pause 
\item (Aside: what does this even mean? It may be for a threshold $\tau$ and points $a, b, c$ that $\dist(a, b) \leqs \kappa$ and $\dist(b, c) \leqs \kappa$ but $\dist(a, c) > \kappa$ i.e. similarity is not ``transitive'').
\end{itemize}
\end{frame}

\begin{frame}
\frametitle{Similarity is not transitive}
\begin{itemize}
\item A more meaningful formulation can be: we want \tcb{all} {\em pairs} of items $a, b$ such that $\si(a, b)\geqs \tau$ for some $\tau \in [0, 1]$ (sufficiently similar items).
\pause
\item If we make $\tau = 0$ we are asking for all pairs of items; this would involve all of the $\binom{n}{2}$ number of pairs (for $n$ items in the database).
\pause
\item Imagine if the number of items $n$ were \pause $1$ million... \pause $1$ billion...!
\end{itemize}
\end{frame}

\begin{frame}
\frametitle{Alternate Formulation/Relaxations}
\begin{itemize}
\item We really do not want \tcb{all} pairs for $\tau$ really small; in fact, typical use-cases will consider $\tau > 0.8$ or so (sufficiently similar).
\item Let's also relax the \tcb{all} in the above too; replace that by \tcr{most}.
\end{itemize}
\end{frame}

\begin{frame}
\frametitle{Refining what we want}
\begin{itemize}
\item Find near-duplicates of query items.
\item Some mistakes will be allowed (both false positives, false negatives). 
\item Time! Querying should be \tcg{fast}. The clustering variant should take $O(n)$ time instead of $O(n^2)$. 
\end{itemize}
\end{frame}


\begin{frame}
\frametitle{Hashing to the rescue}
\begin{itemize}
\item Can we hash items so that ``nearby'' items are in \tcr{same} hash-buckets?
\item Note that this is counter to the usual notion of hashing, where collisions are taboo. 
\item Here, we would like collisions - but only between {\em nearby} items.
\end{itemize}
\end{frame}

\begin{frame}
\frametitle{Enter $\lsh$}
\begin{itemize}
\item Locality Sensitive Hashing - introduced by Indyk \& Motwani, in 1998. 
\begin{itemize}
\item Introduced concept
\item exhibited $\lsh$ for Hamming Distance.
\end{itemize}
\pause
\item MinHash - by Broder, Charikar, Frieze, Mitzenmacher, 1998.
\begin{itemize}
\item Introduced min-wise permutations.
\item $\lsh$ for Jaccard similarity.
\end{itemize}\pause
\item SimHash - by Charikar, 2002. 
\begin{itemize}
\item Demonstrated connections between randomized rounding and $\lsh$
\item $\lsh$ for cosine similarity (angular distance).
\end{itemize}
\end{itemize}
\end{frame}

\begin{frame}
\frametitle{(Slight) Reformulation}
\begin{itemize}
\item A family of hash functions $\mathcal{F}$ is a $\lsh$ if $\Pr_{h\in\mathcal{F}}\left[h(x)=h(y)\right] = \si(x, y)$.
\item Read as: items that are ``highly'' similar land in the \tcb{same} hash-bucket with ``high'' probability, and items that are dissimilar land in the same hash bucket with ``low'' probability.
\end{itemize}
\end{frame}

\begin{frame}
\frametitle{$\lsh$ for angular distance}
\begin{itemize}
\item Given two vectors $a, b$, construct a hash function that maps these to the same bucket if the angle $\theta$ between them is {\em small}.
\item Idea: use a random hyperplane (random projection).
\end{itemize}
\begin{center}
\includegraphics[height=1.5in]{fig1}
\end{center}
\end{frame}

\begin{frame}
\frametitle{$\lsh$ for angular distance}
\begin{itemize}
\item Given two vectors $a, b$, construct a hash function that maps these to the same bucket if the angle $\theta$ between them is {\em small}.
\item Idea: use a random hyperplane (random projection).
\end{itemize}
\begin{center}
\includegraphics[height=1.5in]{fig1}
\end{center}
\end{frame}

\begin{frame}
\frametitle{$\lsh$ for angular distance}
\begin{itemize}
\item Given two vectors $a, b$, construct a hash function that maps these to the same bucket if the angle $\theta$ between them is {\em small}.
\item Idea: use a random hyperplane (random projection).
\end{itemize}
\begin{center}
\includegraphics[height=1.5in]{fig2}
\end{center}
\end{frame}

\begin{frame}
\frametitle{$\lsh$ for angular distance}
\begin{itemize}
\item Given two vectors $a, b$, construct a hash function that maps these to the same bucket if the angle $\theta$ between them is {\em small}.
\item Idea: use a random hyperplane (random projection).
\end{itemize}
\begin{center}
\includegraphics[height=1.5in]{fig3}
\end{center}
\end{frame}

\begin{frame}
\frametitle{$\lsh$ for angular distance}
\begin{itemize}
\item This random hyperplane is a hash function $h: \mathbb{R}^n \rightarrow \{0, 1\}$. 
\item It maps a vector $v$ to $\{0, 1\}$, depending on whether the vector $v$ is to the top/bottom of the hyperplane (essentially, checking for the sign of the inner product).
\end{itemize}
\end{frame}

\begin{frame}
\frametitle{Probability calculation}
\begin{itemize}
\item Probability that vectors $a, b$ map to the same output, i.e. $\Pr\left[h(a) = h(b)\right]$?
\pause
\item If $\theta$ be the angle between $a, b$, then this equals $\left[1 - \frac{\theta}{\pi}\right]$.
\end{itemize}
\end{frame}

\begin{frame}
\frametitle{Precision? Recall?}
\begin{itemize}
\item With a single hash function $h$, the precision may be quite low; the recall quite high. 
\item Just one hash may not be able to detect vectors that are indeed close by. So...
\pause
\item Several (independent) copies of the hash function $h$; call these $h_1, h_2, \cdots, h_k$.  
\end{itemize}
\end{frame}

\begin{frame}
\frametitle{Precision? Recall?}
\begin{itemize}
\item With $k$ hash functions, we have a $k$-bit binary string, corresponding to a vector $v$.
\item Call this the \textit{hashcode} corresponding to the vector $v$ (denoted as $\hc(v)$). 
\item For two vectors $a, b$: if $\hc(a) = \hc(b)$ bit-by-bit, then surely $a$ and $b$ are close. 
\pause
\item This would help in improving precision. But recall may suffer.
\end{itemize}
\end{frame}

\begin{frame}
\frametitle{Precision? Recall? Two extremes}
\begin{itemize}
\item Given the hashcodes of the vectors, we can ask for a full bit-by-bit match to declare near duplicates. High precision, low recall.
\item Given the hashcodes, we can ask for a \textit{single} bit match to declare near duplicates. High recall, low precision.
\pause
\item Mix the two: Banding. In the parlance of complexity theory, \tcr{gap amplification}.
\end{itemize}
\end{frame}

\begin{frame}
\frametitle{Banding}
\begin{itemize}
\item An example with $k = 15$ hashes, where $\mathrm{rows} = 3$ and $\mathrm{bands} = 5$. 
\end{itemize}
\begin{center}
\includegraphics[height=1.5in]{fig4}
\end{center}
\end{frame}

\begin{frame}
\frametitle{Banding}
\begin{itemize}
\item An example with $k = 15$ hashes, where $\mathrm{rows} = 3$ and $\mathrm{bands} = 5$. 
\end{itemize}
\begin{center}
\includegraphics[height=1.5in]{fig5}
\end{center}
\end{frame}

\begin{frame}
\frametitle{Banding}
\begin{itemize}
\item An example with $k = 15$ hashes, where $\mathrm{rows} = 3$ and $\mathrm{bands} = 5$. 
\end{itemize}
\begin{center}
\includegraphics[height=1.5in]{fig6}
\end{center}
\end{frame}

\begin{frame}
\frametitle{Are $a$ and $b$ near-duplicates?}
\begin{itemize}
\item We show the $\hc$'s for $a$ and $b$:
\end{itemize}
\begin{center}
\includegraphics[height=2.5in]{fig7}
\end{center}
\end{frame}

\begin{frame}
\frametitle{Are $a$ and $b$ near-duplicates?}
\begin{itemize}
\item We show the $\hc$'s for $a$ and $b$:
\end{itemize}
\begin{center}
\includegraphics[height=2.5in]{fig8}
\end{center}
\end{frame}

\begin{frame}
\frametitle{Are $a$ and $b$ near-duplicates?}
\begin{itemize}
\item We show the $\hc$'s for $a$ and $b$:
\end{itemize}
\begin{center}
\includegraphics[height=2.5in]{fig9}
\end{center}
\end{frame}


\begin{frame}
\frametitle{Still a flourishing area...}
\begin{itemize}
\item Various considerations:
\begin{itemize}
\item Factors to consider: hash table construction time, query times, etc. 
\item How much randomness do I need?
\item How do I improve recall while maintaining precision.
\item Deep Hashing techniques?
\end{itemize}
\item Improve \textit{recall} of $\lsh$: 
\begin{itemize}
\item covering $\lsh$ (only for Hamming space),
\item other similarity measures wide open.
\end{itemize} 
\item Develop $\lsh$'s for other similarity measures.
\begin{itemize}
\item Inner product 
\item Also gave rise to Assymetric $\lsh$.
\end{itemize}
\item Improve training and query times based on data:
\begin{itemize}
\item Data Dependent Hashing
\item Learning to hash
\end{itemize}
\end{itemize}
\end{frame}

\begin{frame}
\frametitle{}
\begin{itemize}
\item 
\end{itemize}
\end{frame}

\begin{frame}
\begin{center}
Thank You!
\end{center}
\end{frame}

\begin{frame}
\frametitle{Formal definition of $\lsh$}
\begin{itemize}
\item A hash is said to be a $(S, cS, p_1, p_2)$-$\lsh$ for a similarity function $\si$ over the space 
$\mathcal{X}$ if for any $x, y\in\mathcal{X}$:
\begin{itemize}
\item if $\si(x, y)\geqs S$ then $\Pr[h(x)=h(y)]\geqs p_1$.
\item if $\si(x, y)\leqs cS$ then $\Pr[h(x)=h(y)]\leqs p_2$.
\end{itemize}
Here, $c \in (0, 1)$ \pause
\item Read as: items that are ``highly'' similar land in the \tcb{same} hash-bucket with ``high'' probability, and items that are dissimilar land in the same hash bucket with ``low'' probability.
\item Ideally we want: $p_1 = 1$, and $p_2 = 0$. These probabilities ``mirror'' the similarity function $\si(\cdot, \cdot)$.
\end{itemize}
\end{frame}

\end{document} 

\begin{frame}
\frametitle{}
\begin{itemize}
\item
\end{itemize}
\end{frame}
